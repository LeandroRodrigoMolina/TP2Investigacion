\documentclass[]{article}

\usepackage[style=apa, backend=biber, natbib=true, language=spanish, url=true]{biblatex}
\usepackage{hyperref}

\addbibresource{referencias.bib}

%opening
\title{Trabajo practico 2 investigación}
\author{Borgo Martin, Reniero Isaias, Molina Leandro}
\begin{document}

\maketitle
\begin{enumerate}
	\item Buscar en la web el siguiente libro: \\
	Dawson, C. W. (2005). Projects in computing and information systems: a student’s guide. Pearson Education. \\
	Si está disponible como un recurso electrónico, indicar si es posible descargarlo y la fuente de donde lo hicieron.
	\begin{itemize}
		\item No está disponible como recurso electrónico, solo está citado en un par de artículos, 5 versiones y artículos relacionados (por medio de google scholar).
	\end{itemize}
	
	\item Indicar si existen libros de “Métodos de Investigación” los cuales hayan sido publicados desde el 2015 en adelante. Se deberán indicar no más de 3 referencias bibliográficas siguiendo la norma APA.
	\begin{itemize}
		\item Miranda, F. A. (2016). Métodos de investigación histórica. Síntesis.
		\item Pallás, J. M. A., \& Villa, J. J. (2019). Métodos de investigación clínica y epidemiológica. Elsevier Health Sciences
		\item Páramo Reales, D., Sierra, C., Jesús, S., \& Maestre Matos, L. M. (2020). Métodos de investigación cualitativa. Fundamentos y aplicaciones. Editorial Unimagdalena.
	\end{itemize}
	
	\item Seleccionar y buscar 3 trabajos citados en el paper del Trabajo Práctico Nº 1. Indicar si es posible acceder y descargarlos. Indicar el sitio web de donde han tenido acceso al paper.
	\begin{itemize}
		\item Scorzo, R., Favieri, A., \& Williner, B. (2018). Desarrollo de un espacio de enseñanza aprendizaje para realizar actividades con uso de software en una cátedra numerosa. Revista Iberoamericana de Tecnología en Educación y Educación en Tecnología, (21), 77-83. \\
		Este artículo está disponible en \href{https://www.scielo.org}{Scielo}, no permite su descarga sino que posee una visión online (aunque se puede imprimir la página), \href{http://www.scielo.org.ar/scielo.php?pid=S1850-99592018000100010&script=sci_arttext}{clic acá para ver}.
		\item Burbules, N. (2012). El aprendizaje ubicuo y el futuro de la enseñanza Encounters/Encuentros/Rencontres on Education. \href{https://doi.org/10.15572/ENCO2012.01}{clic aca para ver} \\
		Este artículo está disponible y se encuentra para descargar en \href{https://www.researchgate.net}{ResearchGate}, \href{https://www.researchgate.net/publication/287453889_El_aprendizaje_ubicuo_y_el_futuro_de_la_ensenanza}{clic acá para ver}.
		\item Ong, W. J., \& Hartley, J. (2016). Oralidad y escritura: tecnologías de la palabra. Fondo de cultura económica. \\
		Este libro se encuentra disponible para leer en forma online únicamente en \href{https://books.google.com}{Google Books}, aunque también se puede comprar. \href{https://books.google.com.ar/books?id=E5U-DQAAQBAJ&lpg=PT6&ots=_zVsyyZ7Ni&dq=Oralidad%20y%20Escritura&lr&hl=es&pg=PT6#v=onepage&q=Oralidad%20y%20Escritura&f=false}{Clic acá para ver}.
		
	\end{itemize}

	    \item Los Handbooks se caracterizan por ser un excelente recurso para encontrar información sobre una amplia gama de temas de investigación en Ciencias de la Computación, por ejemplo:
	\begin{itemize}
		\item Robinson, A. J., and Voronkov, A. (Eds.). (2001). Handbook of automated reasoning (Vol. 1). Elsevier.
		\item Van Leeuwen, J. (Ed.). (1991). Handbook of theoretical computer science (vol. A) algorithms and complexity. Mit Press.
	\end{itemize}
	¿Algunos de estos Handbooks se encuentran disponibles para descargar? En caso de que no se pueda, ¿Qué información acerca de ellos encuentra en la web?.
	\begin{itemize}
		\item En el primer handbook en Google Scholar encontramos varias citas y el libro para leer en forma online (sin permitir la descarga). En DBLP encontramos el handbook, pero a diferencia de Google Scholar, solo nos brinda el índice del libro (los capítulos). En Science Direct podemos ver detalles del libro como el ISBN, la imprenta y demás información.
		
		\item En el segundo handbook no encontramos en ningún lugar la posibilidad de descargar, solo encontramos citas al libro. Entrando al link que nos proporciona Google Scholar se nos muestra una pequeña review del libro.
	\end{itemize}
	
	\item DBLP, Scopus y Google Scholar son algunos de los más grandes repositorios bibliográficos que se encuentran en la web. Buscar en estos repositorios las publicaciones del autor Christian W. Dawson, e indicar su última publicación.
	\begin{itemize}
		\item Tanto en Google Scholar como en Scopus, la última publicación que nos proveen es la misma, siendo esta de \href{https://scholar.google.com/citations?view_op=view_citation&hl=es&user=QGP7CAsAAAAJ&sortby=pubdate&citation_for_view=QGP7CAsAAAAJ:5ugPr518TE4C}{julio de 2023}: \\
		Wilby, R. L., Dawson, C. W., Yu, D., Herring, Z., Baruch, A., Ascott, M. J., Finney, D. L., Macdonald, D. M. J., Marsham, J. H., Matthews, T., \& Murphy, C. (2023). Spatial and temporal scaling of sub-daily extreme rainfall for data sparse places. Climate Dynamics, 60(11-12), 3577-3596. \href{https://doi.org/10.1007/s00382-022-06528-2}{https://doi.org/10.1007/s00382-022-06528-2}.
		
		\item En el caso de DBLP, solamente tiene publicaciones hasta la fecha del 7 de julio de 2020, cuya última publicación es: \\
		Gazzawe, F., Lock, R., \& Dawson, C. (2020). Traceability framework for requirement artefacts. In Intelligent Computing: Proceedings of the 2020 Computing Conference, Volume 1 (pp. 97-109). Springer International Publishing.
	\end{itemize}

	\item Redactar el tema de investigación de interés seleccionado en una frase. 
	\begin{itemize}
		\item Testeo de eficiencia de algoritmos de criptografía ligera implementados por software basados en block cipher. \\ Sobre criptografía ligera una descripción, y comparar dos algoritmos block ciphers.	
	\end{itemize}
	% Continúa con los otros puntos de la misma manera.
	
	\item Realizar una revisión bibliográfica y elaborar un resumen, de no menos de 2 carillas, del estado actual del conocimiento sobre el tema elegido, es decir describir la situación actual del tema abordado, lo que se conoce y lo que no, lo escrito y lo no escrito, lo evidente y lo tácito.
\end{enumerate}
\printbibliography

\end{document}
